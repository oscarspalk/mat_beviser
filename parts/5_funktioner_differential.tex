
\section{Funktioner og differentialregning}

Forklar hvad det betyder, at en funktion er differentiabel. 

\subsection{Bevis af produktreglen}

\begin{proofw}
    

Vi har 2 funktioner $f$ og $g$, og vi vil finde den afledte af deres produktfunktion, dvs.:

$$
    (f\cdot g)'(x)
$$

For hver funktion kan vi opstille en differenskvotient, som bliver til differentialkvotienter, når vi lader $h \rightarrow 0$:

$$
    \frac{f(x+h)-f(x)}{h} \xrightarrow[h \rightarrow 0]{} f'(x)
$$

$$
    \frac{g(x+h)-g(x)}{h} \xrightarrow[h \rightarrow 0]{} g'(x)
$$

Så opskriver vi differenskvotienten for funktionen $(f \cdot g)(x)$:

$$
    \frac{
        f(x+h)\cdot g(x+h)
        -
        f(x) \cdot g(x)
    }{h}
$$

Så lægger vi $f(x) \cdot g(x+h)$ til og trækker det fra:

$$
    \frac{
        f(x+h)\cdot g(x+h)
        -
        f(x) \cdot g(x+h)
        +
        f(x) \cdot g(x+h)
        -
        f(x) \cdot g(x)
    }{h}
$$

Det ses, at $g(x+h)$ og $f(x)$ kan sættes udenfor parentes:

$$
    \frac{
        g(x+h) \cdot (f(x+h)
        -
        f(x))
        +
        f(x) \cdot (g(x+h)
        -
     \cdot g(x))
    }{h}
$$

Vi opsplitter brøken i 2, og sætter $g(x+h)$ og $f(x)$ udenfor brøkerne:

$$
    g(x+h) \cdot \frac{
          f(x+h)
        -
        f(x)   
    }{h}
    +
        f(x) \cdot 
        \frac{g(x+h)
        -
      g(x)}{h}
$$

Så lader vi $h \rightarrow 0$:

$$
    g(x) \cdot f'(x)+f(x) \cdot g'(x)
$$

Så:

$$
    (f \cdot g)'(x)=    g(x) \cdot f'(x)+f(x) \cdot g'(x)
$$

\end{proofw}
