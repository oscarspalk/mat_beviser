
\section{Kombinatorik og binomialfordelingen}

Forklar hvad der kendetegner et binomialeksperiment, og argumenter for formlen for binomialsandsynligheder. 

\subsection{Bevis for en binomialfordelt stokastisk variabels middelværdi}

\begin{proofw}
    
Vi vil vise, at for $X \sim B(n,p)$, så er $\mu=n\cdot p$.
Vi kan udregne sandsynligheden for $r$ successer ved:

$$
    P(X=r)=\begin{pmatrix}
        n \\ r
    \end{pmatrix}
    \cdot p^r \cdot (1-p)^{n-r}
$$

$X$ kan kun antage hele antal successer, derfor må
middelværdien være summen af hvert antal successer gange sandsynligheden for det antal successer:

$$
\mu =\sum_{r=0}^{n} r\cdot P(X=r)=
\sum_{r=0}^{n} r \cdot \begin{pmatrix}
        n \\ r
    \end{pmatrix}
    \cdot p^r \cdot (1-p)^{n-r}
$$

Da vi ganger med $r$, så er iterationen af $r=0$ ligegyldig, da den ikke bidrager til summen,
derfor indekserer vi fra $r=1$:

$$
\mu =
\sum_{r=1}^{n} r \cdot \begin{pmatrix}
        n \\ r
    \end{pmatrix}
    \cdot p^r \cdot (1-p)^{n-r}
$$

Nu betragter vi $r \cdot \begin{pmatrix}
        n \\ r
    \end{pmatrix}$ leddet, hvilket vi kan omskrive:

$$
    r \cdot \begin{pmatrix}
        n \\ r
    \end{pmatrix}
    =
    r \cdot \frac{n!}{r!(n-r)!}
    =
    r \cdot \frac{1 \cdot 2 \cdot 3 ... n}{
        1 \cdot 2 \cdot 3...r(n-r)!
    }
    =n \cdot \frac{(n-1)!}{
        (r-1)!(n-r)!
    }
    =
    n \cdot \frac{(n-1)!}{
        (r-1)!((n-1)-(r-1))!
    }
$$

Vi indfører $r'=r-1$ og $n'=n-1$, så udtrykket bliver til:

$$
r \cdot \begin{pmatrix}
        n \\ r
    \end{pmatrix}
    =
    n \cdot \frac{n'!}{r'!(n'-r')!}
    =
    n \cdot \begin{pmatrix}
        n' \\ r'
    \end{pmatrix}
$$

Det kan vi nu indsætte i vores sum:

$$
\mu =
\sum_{r=1}^{n}  n \cdot \begin{pmatrix}
        n' \\ r'
    \end{pmatrix}
    \cdot p^r \cdot (1-p)^{n-r}
$$

Da $n$ er en konstant, så kan vi sætte den udenfor og vi kan sætte et $p$ udenfor,
og i det sidste led må $n-r=n'-r'$, derfor:

$$
\mu =
n \cdot p \cdot \sum_{r=1}^{n}  \begin{pmatrix}
        n' \\ r'
    \end{pmatrix}
    \cdot p^{r'} \cdot (1-p)^{n'-r'}
$$

Da hele udtrykket består af $r'$ og $n'$, så anvender vi disse i sumoperatøren i stedet:

$$
\mu =
n \cdot p \cdot \sum_{r'=0}^{n'}  \begin{pmatrix}
        n' \\ r'
    \end{pmatrix}
    \cdot p^{r'} \cdot (1-p)^{n'-r'}
$$

Betragter vi en anden stokatisk variabel $Z \sim B(n', p)$,
må sandsynligheden for $r'$ antal successer være:

$$
    P(Z=r')=\begin{pmatrix}
        n' \\ r'
    \end{pmatrix}
    \cdot p^{r'}
    \cdot (1-p)^{n'-r'}
$$

Summen vi tager, som lægger alle sandsynligheder sammen
må derfor give 1, og deraf:

$$
\mu =
n \cdot p \cdot 1=n\cdot p
$$

\end{proofw}
