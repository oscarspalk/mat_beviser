
\section{Differentialligninger}

Forklar betydning af differentialligning, samt hvad det betyder at en funktion er en løsning til en differentialligning.

\subsection{Bevis af panserformlen}

\begin{proofw}
    
Vi vil vise den fuldstændige løsning til differentialligningen

$$
    y'+a(x)\cdot y=b(x) \Leftrightarrow y'=-a(x) \cdot y + b(x) 
$$

Vi introducerer en hjælpe funktion $z$:

$$
    z(x)=f(x) \cdot e^{A(x)}
$$

Som vi differentierer, vi udnytter at pga. kædereglen,
så er nedenstående sandt:

$$
    \left(e^{A(x)}\right)'=a(x) \cdot e^{A(x)}
$$

Og vi husker at anvende produktreglen:

$$
    z'(x)=f(x) \cdot a(x) \cdot e^{A(x)}+f'(x)\cdot e^{A(x)}
$$

Vi indsætter differentialligningen og sætter $e^{A(x)}$ udenfor parentes:

$$
    z'(x)= e^{A(x)} \cdot (f(x) \cdot a(x) +f'(x))=
    e^{A(x)} \cdot (f(x) \cdot a(x) +b(x)-a(x)\cdot f(x))
    = b(x) \cdot e^{A(x)}
$$

Da vores afledte kun er bestemt med udtryk af $x$,
så kan vi blot integrere den for at finde stamfunktionen:

$$z(x)= \int b(x) \cdot e^{A(x)} \,dx+c$$

Nu kan vi isolere $f(x)$ fra vores første $z(x)$ udtryk:

$$
    f(x) \cdot e^{A(x)}=\int b(x) \cdot e^{A(x)} \,dx+c
$$

Vi deler med $e^{A(x)}$, hvilket svarer til at gange med $e^{-A(x)}$:

$$
    f(x)=e^{-A(x)} \cdot \int b(x) \cdot e^{A(x)} \,dx+c \cdot e^{-A(x)}
$$

Så er panserformlen bevist.

\end{proofw}
