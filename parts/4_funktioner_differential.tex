\section{Funktioner og differentialregning}

\emph{Forklar hvad det betyder, at en funktion er differentiabel.}

En funktion er defineret i et interval, i et mindre udsnit
eller et mindre interval definerer vi hældningen
som ændringen på $y$-aksen over ændringen på $x$-aksen.
For store $\Delta x$ er det ret grove approksimationer,
men når $\Delta x \rightarrow 0$, så er hældningen
et godt udtryk for, hvor meget funktionen vokser lige nu.

Det kræver dog, at funktionen er defineret i intervallet
og er kontinuer, altså ikke knækker midt i det hele.

Hvis en funktion er det, så kan vi opskrive dens differenskvotient:

$$
    \frac{\Delta y}{h}=\frac{f(x+h)-f(x)}{h}
$$

Hvis vi lader $h \rightarrow 0$, så får vi differentialkvotienten, som vi noterer:

$$
    \lim_{h \rightarrow 0}     \frac{\Delta y}{h}=\frac{f(x+h)-f(x)}{h}
    =f'(x)
$$

Den naturlige eksponential funktion er defineret som Eulers tal $e$ opløftet i $x$:

$$f(x)=e^x$$

Den har samme egenskab som alle andre eksponentielle funktioner, at den går gennem $(0,1)$.
Det som gør den speciel er, at dens hældning også er $1$ i $(0,1)$.

\subsection{Bevis af differentialkvotienten for den naturlige eksponentialfunktion}
\begin{proofw}
    Vi anvender, at $f'(0)=1$, da vi kan opskrive differenskvotienten i $0$ som:

    $$
    \lim_{h \rightarrow 0}
    \frac{e^h-1}{h}=1
    $$

    Vi kan også opskrive differenskvotienten for et vilkårligt punkt:

    $$
        \frac{\Delta y}{h}=\frac{e^{x_0+h}-e^{x_0}}{h}=e^{x_0}\cdot \frac{e^h-1}{h}
    $$

    Når vi lader $h \rightarrow 0$, så:

    $$
       f'(x)= \lim_{h \rightarrow 0} e^{x_0} \cdot \frac{e^h-1}{h}=e^{x_0} \cdot 1=e^{x_0}
    $$

    Derved er det bevist, at $\frac{\partial}{\partial x} e^x=e^x$.
\end{proofw}
