\documentclass{article}
    % General document formatting
    \usepackage[margin=0.7in]{geometry}
    \usepackage{amsthm}
    \usepackage[parfill]{parskip}
    \usepackage[utf8]{inputenc}
    \usepackage{cancel}
    \usepackage{graphicx}
    \usepackage[danish]{babel}
    \usepackage{mathtools}

    \graphicspath{./skitser/}
    % Related to math
    \usepackage{amsmath,amssymb,amsfonts,amsthm}

\makeatletter
\newenvironment{proofw}{\par
  \pushQED{\qed}%
  \normalfont \topsep6\p@\@plus6\p@\relax
  \trivlist
  \item[]\ignorespaces
}{%
  \popQED\endtrivlist\@endpefalse
}
\makeatother

\begin{document}

\tableofcontents

\section{Vektorer og trigonometri}

\subsection{Bevis}

\begin{proofw}
    
Betragt figur \ref{fig:trekant_vektor}, hvor en vinkel $v$ er udspændt af vektoren $\vec{x}$ og $\vec{y}$.

\begin{figure}[h]
    \centering
    \includegraphics[scale=0.3]{./skitser/trekant_vektor_skitse.png}
    \label{fig:trekant_vektor}
    \caption{Vinkel udspændt af 2 vektorer.}
\end{figure}

Vha. af disse vektorer kan vi danne en trekant, hvori vi kan anvende cosinusrelationerne
til at lave et generelt udtryk for vinklen udspændt af 2 vektorer.
Her kan vi lave et udtryk for den lange side ved at anvende indskudsreglen:

\begin{align*}
    \vec{AB}+\vec{BC}&=\vec{AC}
    \\
    &\Downarrow
    \\
    \vec{BC}&=\vec{AC}-\vec{AB}
    \\
    &\Downarrow
    \\
    \vec{CB}&=-\vec{BC}=\vec{AB}-\vec{AC}
\end{align*}

Så den trejde side noteres $\vec{x}-\vec{y}$, og vi anvender cosinusrelationen, der siger, at i en trekant, så:

$$
    c^2=a^2+b^2-2ab \cdot \cos(v)
$$

Hvilket i vores tilfælde betyder:

$$
    |\vec{x}-\vec{y}|^2=|\vec{x}|^2+|\vec{y}|^2-2|\vec{x}||\vec{y}| \cdot \cos(v)
$$

Nu anvendes det, at $|\vec{x}|^2=\vec{x} \cdot \vec{x}$, det betyder for vores vektor:

$$
    |\vec{x}-\vec{y}|^2=(\vec{x}-\vec{y})\cdot (\vec{x}-\vec{y})
    =|\vec{x}|^2+|\vec{y}|^2-2 \cdot \vec{x}\cdot \vec{y}
$$

Det indsættes i ovenstående:

$$
\cancel{|\vec{x}|^2+|\vec{y}|^2-2} \cdot \vec{x}\cdot \vec{y}
=\cancel{|\vec{x}|^2+|\vec{y}|^2-2}|\vec{x}||\vec{y}| \cdot \cos(v)
$$

Hvor $\cos(v)$ isoleres:

$$
    \cos(v)=\frac{
        \vec{x} \cdot \vec{y}
    }{
        |\vec{x}||\vec{y}|
    }
$$
\end{proofw}

\section{Vektorer og linjer i planen}

\subsection{Bevis af linjens parameterfremstilling}

\begin{proofw}
    Betragt følgende skitse:
    \begin{figure}[h]
        \centering
        \includegraphics[scale=0.4]{skitser/linje_parameter.png}
    \end{figure}

Tager vi afsæt i punktet $P_0(x_0,y_0)$, hvis position kan beskrives
med vektoren $\vec{OP}$. Tager vi et skridt langs en vektor, der er parallel med linjen,
så vil vores nye position også være på linjen.
Derfor må vi ved at gange retningsvektoren med et vilkårligt tal
kunne ramme alle punkter på linjen.
Så kan vi beskrive vektoren til punktet $P(x,y)$
som vektoren til $P_0$ plus retningsvektoren skaleret:

$$
\vec{OP}=\vec{OP_0}+t\cdot \vec{r}
$$

Vi opsplitter $\vec{OP}$ i $\begin{pmatrix}
    x \\ y
\end{pmatrix}$ og $\vec{r}$ i $\begin{pmatrix}
    r_1 \\ r_2
\end{pmatrix}$:

$$
\begin{pmatrix}
    x
    \\
    y
\end{pmatrix}
=\begin{pmatrix}
    x_0
    \\
    y_0
\end{pmatrix}
+
t \cdot \begin{pmatrix}
    r_1
    \\
    r_2
\end{pmatrix}
$$
\end{proofw}

\subsection{Bevis af linjens ligning}

\begin{proofw}
    Betragt nedenstående figur:
    \begin{figure}[h]
        \centering
        \includegraphics[scale=0.4]{skitser/linjens_ligning.png}
    \end{figure}

    Vi kender $P_0(x_0,y_0)$ og normalvektoren $\vec{n}=\begin{pmatrix}
        a \\ b
    \end{pmatrix}$, $P(x,y)$ er et vilkårligt punkt langs linjen, som vi vil beskrive.
    Vi kan lave en ortogonal vektor til $\vec{n}$ ved at lave vektoren $\vec{P_0P}=\begin{pmatrix}
        x-x_0
        \\
        y-y_0
    \end{pmatrix}$.
    Tricket er så, at vi har 2 ortogonale vektorer, hvilket betyder,
    at deres skalarprodukt er 0, derved kan vi opstille følgende udtryk, hvor $x$ og $y$ alle er punkter på linjen:

    \begin{align*}
        \vec{P_0P}&\cdot\vec{n}=0
        \\
        &\Downarrow
        \\
        \begin{pmatrix}
            x-x_0
            \\
            y-y_0
        \end{pmatrix}
        &\cdot
        \begin{pmatrix}
            a \\
            b
        \end{pmatrix}=0
        \\
        \Downarrow
        \\
        a(x-x_0)&+b(y-y_0)=0
        \\
        \Downarrow
        \\
        ax+by+&(-ax_0-by_0)=0
        \\
        \Downarrow
        \\
        ax+by&+c=0
    \end{align*}

    Det er vist, at en linje kan beskrives ud fra et punkt og en normalvektor til linjen.

\end{proofw}

\section{Vektorer og vektorfunktioner}

\subsection{Bevis af cirklens parameterfremstilling}

\begin{proofw}
    \begin{figure}[h]
        \centering
        \includegraphics[scale=0.3]{skitser/cirkel.png}
    \end{figure}

    Betragt ovenstående figur, først tager vi tilfældet,
    hvor cirklen har centrum i origo. Her vil alle punkter
    på cirkelperifirien kunne beskrives som en skalering af enhedscirklen, derfor:

    $$
        \vec{OP}=\begin{pmatrix}
            r\cdot \cos(t)
            \\
            r\cdot \sin(t)
        \end{pmatrix}
    $$
    
    For cirklen, der ikke har centrum i origo, er situationen en smule anderledes,
    men vektoren $\vec{CQ}=\vec{OP}$, da cirklerne har samme radius $r$.
    Vi anvender indskudsreglen til at finde:

    $$
        \vec{OQ}=\vec{OC}+\vec{CQ}
    $$

    Disse værdier kender vi, så parameterfremstillingen bliver:

    $$
    \begin{pmatrix}
        x \\ y
    \end{pmatrix}
    =\begin{pmatrix}
        a \\ b
    \end{pmatrix}
    +
    \begin{pmatrix}
        r \cdot \cos(t)
        \\
        r \cdot \sin(t)
    \end{pmatrix}
    =\begin{pmatrix}
        a+ r \cdot \cos(t)
        \\
        b+ r \cdot \sin(t)
    \end{pmatrix}
    $$

\end{proofw}

\section{Funktioner og differentialregning}

Den naturlige eksponential funktion er defineret som Eulers tal $e$ opløftet i $x$:

$$f(x)=e^x$$

Den har samme egenskab som alle andre eksponentielle funktioner, at den går gennem $(0,1)$.
Det som gør den speciel er, at dens hældning også er $1$ i $(0,1)$.

\subsection{Bevis af differentialkvotienten for den naturlige eksponentialfunktion}
\begin{proofw}
    Vi anvender, at $f'(0)=1$, da vi kan opskrive differenskvotienten i $0$ som:

    $$
    \lim_{h \rightarrow 0}
    \frac{e^h-1}{h}=1
    $$

    Vi kan også opskrive differenskvotienten for et vilkårligt punkt:

    $$
        \frac{\Delta y}{h}=\frac{e^{x_0+h}-e^{x_0}}{h}=e^{x_0}\cdot \frac{e^h-1}{h}
    $$

    Når vi lader $h \rightarrow 0$, så:

    $$
       f'(x)= \lim_{h \rightarrow 0} e^{x_0} \cdot \frac{e^h-1}{h}=e^{x_0} \cdot 1=e^{x_0}
    $$

    Derved er det bevist, at $\frac{\partial}{\partial x} e^x=e^x$.
\end{proofw}

\section{Funktioner og differentialregning}

\subsection{Bevis af produktreglen}

\begin{proofw}
    

Vi har 2 funktioner $f$ og $g$, og vi vil finde den afledte af deres produktfunktion, dvs.:

$$
    (f\cdot g)'(x)
$$

For hver funktion kan vi opstille en differenskvotient, som bliver til differentialkvotienter, når vi lader $h \rightarrow 0$:

$$
    \frac{f(x+h)-f(x)}{h} \xrightarrow[h \rightarrow 0]{} f'(x)
$$

$$
    \frac{g(x+h)-g(x)}{h} \xrightarrow[h \rightarrow 0]{} g'(x)
$$

Så opskriver vi differenskvotienten for funktionen $(f \cdot g)(x)$:

$$
    \frac{
        f(x+h)\cdot g(x+h)
        -
        f(x) \cdot g(x)
    }{h}
$$

Så lægger vi $f(x) \cdot g(x+h)$ til og trækker det fra:

$$
    \frac{
        f(x+h)\cdot g(x+h)
        -
        f(x) \cdot g(x+h)
        +
        f(x) \cdot g(x+h)
        -
        f(x) \cdot g(x)
    }{h}
$$

Det ses, at $g(x+h)$ og $f(x)$ kan sættes udenfor parentes:

$$
    \frac{
        g(x+h) \cdot (f(x+h)
        -
        f(x))
        +
        f(x) \cdot (g(x+h)
        -
     \cdot g(x))
    }{h}
$$

Vi opsplitter brøken i 2, og sætter $g(x+h)$ og $f(x)$ udenfor brøkerne:

$$
    g(x+h) \cdot \frac{
          f(x+h)
        -
        f(x)   
    }{h}
    +
        f(x) \cdot 
        \frac{g(x+h)
        -
      g(x)}{h}
$$

Så lader vi $h \rightarrow 0$:

$$
    g(x) \cdot f'(x)+f(x) \cdot g'(x)
$$

Så:

$$
    (f \cdot g)'(x)=    g(x) \cdot f'(x)+f(x) \cdot g'(x)
$$

\end{proofw}

\section{Funktioner i to variable og differentialregning}

\subsection{Bevis af tangentplanens ligning}

\begin{proofw}

Betragt følgende skitse:

\begin{figure}[h]
    \centering
    \includegraphics[scale=0.5]{skitser/tangent_plan.png}
\end{figure}

På skitsen er et plan, der følger linjerne $l$ og $m$.
Noterer vi hældningen af $l$ som $p$ og hældningen af $m$ som $q$,
så kan vi opstille følgende udtryk for ændringen på $z$-aksen:

$$
    \Delta z_1=p \cdot (x-x_0)
$$

$$
    \Delta z_2=q \cdot (y-y_0)
$$

Derfor må den nye funktionsværdi $z$ i punktet $(x,y,z)$ være:

$$
    z=z_0+\Delta z_1+\Delta z_2=p \cdot (x-x_0)+q \cdot (y-y_0) + z_0
$$

Det vil sige, at alle $(x,y,z)$, der gør nedenstående ligning sand, er punkter i planet tilhørende
linje $l$ og $m$ med afsæt i punkt $(x_0,y_0,z_0)$:

$$
z=p \cdot (x-x_0)+q \cdot (y-y_0) + z_0
$$

Det ovenstående er dog blot for det generelle plan.
For en tangent plan, som viser hældningen af en funktion af 2 variable,
kan hældningen langs $x$-aksen beskrives som $f_x'(x_0,y_0)$
og langs $y$-aksen $f_y'(x_0,y_0)$.
Sidst kan $z_0$ beskrives som $f(x_0,y_0)$, derfor:

$$
    z=f_x'(x_0,y_0)(x-x_0)+
    f_y'(x_0,y_0)(y-y_0)+
    f(x_0,y_0)
$$

Dette er ligningen for tangentplanet for en funktion af 2 variable.

\end{proofw}

\section{Integralregning og stamfunktioner}

\subsection{Bevis af at alle stamfunktioner til $f(x)$ er på form $F(x)+k$}

\begin{proofw}
    Først vises det, at $F(x)+k$ er en stamfunktion til $f(x)$ da:

    $$
        (F(x)+k)'=F'(x)+k'=f(x)
    $$

    Så vil vi vise, at en anden funktion $G(x)=F(x)+k$ også er stamfunktion,
    da differensfunktionen mærket er 0:

    $$
        (G(x)-F(x))'=F'(x)+k'-F'(x)=f(x)-f(x)=0
    $$

    Derfor:

    $$
        G(x)-F(x)=k \Leftrightarrow G(x)=F(x)+k
    $$

\end{proofw}

\section{Integralregning og arealer}

\subsection{Bevis af at arealfunktionen er stamfunktionen}

\begin{proofw}
    
Betragt følgende skitse:

\begin{figure}[h]
    \centering
    \includegraphics[scale=0.3]{skitser/areal_funktion.png}
\end{figure}

Vi betragter et udsnit af en funktion,
hvor vi ønsker at finde arealet mellem funktionen og $x$-aksen.
Vi antager, at en funktion $A(x)$ giver arealet under grafen indtil $x$-værdien,
og at arealet under grafen på udsnittet ville være givet ved:

$$
    A_{graf}=A(x+h)-A(x)
$$

I udsnittet har vi markeret et interval $x$ til $x+h$,
indenfor dette interval er der et lokalt minimum og maksimum for funktionen.
Arealet under grafen skal være større end eller lig arealet af den blå kasse,
som har længde $h$ og højde $f(x_1)$.
Og arealet skal også være mindre eller lig
arealet af den blå kasse + den røde kasse,
hvilket er den kasse, som har længde $h$ og højde $f(x_2)$.
Derfor kan vi opstille følgende ulighed, som vi kan regne med:

$$
    f(x_1) \cdot h \leq A(x+h)-A(x)
    \leq f(x_2) \cdot h
$$

Først dividerer vi med $h$ på alle sider:

$$
    f(x_1) \leq \frac{A(x+h)-A(x)}{h} \leq f(x_2)
$$

Og så lader vi $h \rightarrow 0$ og deraf:

\begin{align*}
    x_1 &\rightarrow x \\
    x_2 &\rightarrow x \\
    f(x_1) &\rightarrow f(x) \\
    f(x_2) &\rightarrow f(x) \\
    \frac{A(x+h)-A(x)}{h} &\rightarrow A'(x)
\end{align*}

Dette vil sige, at:

$$
    f(x) \leq A'(x) \leq f(x)
$$

Så:
$$
    f(x) = A'(x)
$$

Altså er arealfunktionen af en graf en stamfunktion til funktionen.

\end{proofw}

\section{Integralregning og omdrejningslegeme}

\subsection{Bevis af volumen af omdrejningslegeme}



\end{document}
